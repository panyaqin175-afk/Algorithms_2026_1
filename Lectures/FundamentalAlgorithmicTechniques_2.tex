
\documentclass{beamer}
\usepackage[utf8]{inputenc}
\usepackage{url}

\usepackage{lmodern}
\usepackage{natbib}
\usepackage{tikz}
\usepackage{physics}
\usepackage{graphicx}
\graphicspath{ {./images/} }
\usepackage{booktabs}
\usetikzlibrary{mindmap, trees, shadows, shapes, calc, fadings, positioning, decorations.pathreplacing, intersections, shapes, arrows}

\usetheme{default}
\usepackage{graphics}

\usepackage{color}
\definecolor{new_turquoise}{RGB}{40,151,158}
\setbeamercolor{title}{fg=new_turquoise}




\makeatletter
\setbeamertemplate{frametitle}{
    \ifbeamercolorempty[bg]{frametitle}{}{\nointerlineskip}%
    \@tempdima=\textwidth%
    \advance\@tempdima by\beamer@leftmargin%
    \advance\@tempdima by\beamer@rightmargin%
    \vspace*{0.8cm} 
    \begin{beamercolorbox}[sep=0.3cm,center,wd=\the\@tempdima]{frametitle}
        \usebeamerfont{frametitle}%
        \vbox{}\vskip-1ex%
        \if@tempswa\else\csname beamer@ftecenter\endcsname\fi%
        \strut\insertframetitle\strut\par%
        {%
            \ifx\insertframesubtitle\@empty%
            \else%
            {\usebeamerfont{framesubtitle}\usebeamercolor[fg]{framesubtitle}\insertframesubtitle\strut\par}%
            \fi
        }%
        \vskip-1ex%
        \if@tempswa\else\vskip-.3cm\fi
    \end{beamercolorbox}%
}
\makeatother

\setbeamercolor{frametitle}{fg=new_turquoise}
\setbeamertemplate{itemize item}{\color{new_turquoise}$\blacksquare$}
\setbeamertemplate{itemize subitem}{\color{new_turquoise}$\blacksquare$}


\setbeamertemplate{enumerate item}{\color{new_turquoise}\insertenumlabel}
\setbeamertemplate{enumerate subitem}{\color{new_turquoise}\insertsubenumlabel}

\setbeamertemplate{caption}{\raggedright\insertcaption\par}

\setbeamercolor{section in toc}{fg=new_turquoise}
\setbeamercolor{subsection in toc}{fg=new_turquoise}
\setbeamercolor{subsubsection in toc}{fg=new_turquoise}



\usebackgroundtemplate{
    \includegraphics[width=\paperwidth,height=\paperheight]{figs/slide-title.jpg}
} 

\title{\fontsize{49}{7.2}{\bf Fundamental Algorithmic Techniques II}}
\date{\color{new_turquoise}\today}

\begin{document}
\frame{\titlepage}

\usebackgroundtemplate{
    \includegraphics[width=\paperwidth,height=\paperheight]{figs/slide-pages}
} 


\setbeamertemplate{subsection in toc}{
  \color{new_turquoise}$\blacksquare$\color{black}~~\inserttocsubsection
}


\begin{frame}{Outline}
    \tableofcontents
\end{frame}

\section{Definition and Importance of Algorithms}

\begin{frame}
    \frametitle{\bf Name and Definition}


{\bf Definition} \\
Explicit, precise and unambiguous instructions describing mechanically executable sequence to achieve specific purpose.\\
\bigskip
\bigskip 
{\bf Algorithms} are omnipresent in modern world! \\In our understanding of the world! \\


\bigskip 
\bigskip
{\bf Confused Name!}
\begin{itemize}
    \item $\alpha \lambda \gamma o \zeta$ (algos) = "pain", $\alpha \rho \iota \theta \mu o \zeta$ (arithmos)= "number"!?
    \item Muhammad ibn Musa al-Khwarizmi, c.780--c.850\\
    Algebra/Null,  born in/near Kazakhstan
\end{itemize}
Al-Khwarizmi $\longrightarrow$ "Algorism" in medieval Italy \\$\longrightarrow$ {\bf Algorithm} by "correction/confusion"

\end{frame}


\begin{frame}
    \frametitle{\bf  Algorithms Description}

\begin{enumerate}
    \item {\bf What?} Specify the Problem
    \item {\bf How?} Describe Algorithm (Pseudocode and english)
    \item {\bf Why?} Proof (induction, ...)
    \item {\bf Performance?} Analysis (time and space complexity, ...)
    
\end{enumerate}
\bigskip 
\bigskip 
"Thinking and solving Algorithms":
\begin{itemize}
    \item {\bf Healthy, powerful Basis for thinking!}
    \item Basis and support for Communication (1. \& 4.)!
\end{itemize}

\end{frame}

\begin{frame}
\frametitle{\bf Importance of Algorithms for Developer}
{\bf For oneself:}
\begin{itemize}
    \item Solve Problems!
    \item {\bf Toolbox for cleaner, better, faster designs}
    \item Better communication/confidence
    \item Know/intuit what is a good/bad/solvable/unsolvable problem
    \item Helpful working with AI
\end{itemize}
\bigskip
\bigskip
\bigskip

{\bf Industry:}
\begin{itemize}
    \item Way to screen candidates
    \item Performance/costs awareness or optimisation
    \item Communication is key!
\end{itemize}

\end{frame}


\section{Omnipresence and examples of Algorithms}

\begin{frame}
    \frametitle{\bf  Omnipresence  of Algorithms}
Commercial, Fin., Tech., Industry, Economy, \textcolor{new_turquoise}{\bf AI}: 



\begin{figure}
    \centering
\includegraphics[width=3cm,angle=0]{Algos_figs/tot_cover.png}
\includegraphics[width=3cm,angle=0]{Algos_figs/tree_based_experimentation.png}
\includegraphics[width=3cm,angle=0]{Algos_figs/rag.png}
    \caption{Tree Of Thoughts/ Tree based Experimentation for AI Researcher/ RAG}
\end{figure}


\begin{figure}
    \centering
    \includegraphics[width=4cm, height=2cm,angle=0]{Algos_figs/PageRanks-Example.svg.png}
    \includegraphics[width=4cm, height=2cm,angle=0]{Algos_figs/yandex.png}
    \caption{Left: PageRank, Right: Yandex}
\end{figure}

\end{frame}







\begin{frame}
\frametitle{\bf Introductory Examples}
\begin{figure}
    \centering
    \includegraphics[width=4cm,angle=0]{Algos_figs/330px-Rhind_Mathematical_Papyrus.jpg}
    \includegraphics[angle=0,width=4cm]{Algos_figs/Fibo_Lattice.png}
    \caption{Left Rhind Papyrus, Right, Fibonacci Lattice}
    \label{fig:placeholder}
\end{figure}

\begin{itemize}
    \item Multiplication Algorithms: 
        \begin{itemize}
            \item Peasant Multiplication ($\sim 2000 $ BC, Rhind Papyrus)
            \item Fibonacci Lattice $\sim 1600$
        \end{itemize}
    \item Real World Example: naive $N^2$ sorting taking days!
    
\end{itemize}

\end{frame}



\section{Algorithms Analysis}


\begin{frame}
{Algorithmic Performance Model}

Random Access Machine {\bf RAM} model: 

\begin{itemize}
    \item $\approx$ computer independant
    \item each operation take $\approx$ same compute
    \item operation input size $\approx$ independant
    \item input
\end{itemize}
\medskip
{\bf Operations:} 
\begin{enumerate}
    \item $+,-, *, /, ...$
    \item return and comparisons $==, >, <, \geq, \leq, \% , ...$
    \item variables access, allocation or change
\end{enumerate}
!!!Subroutines or iterations are not considered operations!!! \\
\bigskip

\textcolor{new_turquoise}{\bf This is an approximation!!!}
\end{frame}

\begin{frame}
{Full Analysis: Insertion sort}

\begin{figure}
    \centering
    \includegraphics[width=5cm,angle=0]{Algos_figs/insertion_sort.png}
    \includegraphics[angle=0,width=5cm]{Algos_figs/insertion_sort_costs.png}
    \caption{Left: schema, Right: code}
\end{figure}

Full Analysis:
\footnotesize
\begin{align*}
\#(n) = {} & c_1 n + c_2 (n-1) + c_4 (n-1) + c_5 \sum_{i=2}^n t_i + (c_6 + c_7) \sum_{i=2}^n (t_i - 1) + c_8 (n-1) \\
    = {}  & (c_5 + c_6 + c_7)/2 n^2 + (c1 + c2 + c4 + c5 - c6 - c7 + c8)n \\
    - {} & (c2 + c4 + c5 + c8), \quad  \mathrm{with} \,\sum_{i=2}^n t_i =  \frac{n(n+1)}{2} -1, \, \sum_{i=2}^n (t_i-1)  = \frac{n(n-1)}{2} 
\end{align*}
... Tedious to analyse like this...
\end{frame}

\begin{frame}
{Approximating Performance}

\begin{itemize}
    \item Time Complexity: \\
    Best, Worst, and Average-case Complexity 
    \item Space Complexity:
     total memory an algorithm requires to solve a problem
\end{itemize}
\bigskip 
Both quantities vary a lot for some algorithms: \\
$\Rightarrow$ simplify with  assymptotic $\mathcal{O}$ notations.

\end{frame}





\begin{frame}
{$\mathcal{O}$, $\Theta$, $\Omega$ asymptotic Notations}

Ideas: \\
\begin{enumerate}
    \item Routines like functions scaling with the problem size $N: f(N)$
    \item At $N$ large, $ N \ll N\mathrm{log}(N) \ll N^2 \ll ... \ll \exp(N) $
     \\ 
$\Rightarrow$ For approximation, just scale matters!
\end{enumerate}


\begin{itemize}
    \item $\vb{\mathcal{O}}$: {\bf Big O notation}
    \item $\Theta$: upper bound

    \item $\Omega$: lower bound
\end{itemize}

\smallskip
Examples:
\begin{enumerate}
    \item $\#_1(n)=n$, $\#_2(n)=3n +6$, $\#_3(n) = 15n$: $\Theta(n), \mathcal{O}(n)$,$\Omega(n)$. 
    \item $\#(n) = 7n^3 + 100n^2 + 20n $ is $\Theta(n^3), \mathcal{O}(n^3)$, $\Omega(n^2)$
    \item $\#(x) = (13+2+7+1111)n^{45}$ is  $\Theta(n^{45}), \mathcal{O}(n^{45})$, $\Omega(n^{45})$ 
\end{enumerate}

\end{frame}



\begin{frame}
    {Usage of $\mathcal{O}$ notations}

Most of the time, we are interested in $\mathcal{O}$ and use it for:
\begin{itemize}
    \item Algorithm Analysis
    \item Algorithm Comparisons
\end{itemize}

\begin{figure}
\centering
\includegraphics[width=4cm, height=2cm,angle=0]{Algos_figs/dubRF.png}
\includegraphics[width=4cm, height=2cm,angle=0]{Algos_figs/PB0M1.png}
\includegraphics[width=4cm, height=2cm,angle=0]{Algos_figs/fdcL3.png}

\caption{$\mathcal{O}(N)=N^2 \,\, vs \,\, N\mathrm{log}N \,\, vs \,\, N$ as N increases}
\end{figure}

\end{frame}


\begin{frame}
{When/Why $\mathcal{O}(n) \approx log_2(n)$ }
Consider, as for peasant multiplication: 
$$T(n) \approx n/2 $$
For a given n the algorithms decomposition takes m steps:
\[
n = 2^{\text{m}}
\]
\[
\Rightarrow \log_2(n) = \text{m}
\]


\end{frame}

\end{document}