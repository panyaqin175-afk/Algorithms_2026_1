

\documentclass{beamer}
\usepackage[utf8]{inputenc}
\usepackage{url}

\usepackage{amsmath}
\usepackage{lmodern}
\usepackage{natbib}
\usepackage{tikz}
\usepackage{physics}
\usepackage{graphicx}
\graphicspath{ {./images/} }
\usepackage{booktabs}
\usetikzlibrary{mindmap, trees, shadows, shapes, calc, fadings, positioning, decorations.pathreplacing, intersections, shapes, arrows}

\usetheme{default}
\usepackage{graphics}

\usepackage{color}
\definecolor{new_turquoise}{RGB}{40,151,158}
\setbeamercolor{title}{fg=new_turquoise}


\makeatletter
\setbeamertemplate{frametitle}{
    \ifbeamercolorempty[bg]{frametitle}{}{\nointerlineskip}%
    \@tempdima=\textwidth%
    \advance\@tempdima by\beamer@leftmargin%
    \advance\@tempdima by\beamer@rightmargin%
    \vspace*{0.8cm} 
    \begin{beamercolorbox}[sep=0.3cm,center,wd=\the\@tempdima]{frametitle}
        \usebeamerfont{frametitle}%
        \vbox{}\vskip-1ex%
        \if@tempswa\else\csname beamer@ftecenter\endcsname\fi%
        \strut\insertframetitle\strut\par%
        {%
            \ifx\insertframesubtitle\@empty%
            \else%
            {\usebeamerfont{framesubtitle}\usebeamercolor[fg]{framesubtitle}\insertframesubtitle\strut\par}%
            \fi
        }%
        \vskip-1ex%
        \if@tempswa\else\vskip-.3cm\fi
    \end{beamercolorbox}%
}
\makeatother

\setbeamercolor{frametitle}{fg=new_turquoise}
\setbeamertemplate{itemize item}{\color{new_turquoise}$\blacksquare$}
\setbeamertemplate{itemize subitem}{\color{new_turquoise}$\blacksquare$}


\setbeamertemplate{enumerate item}{\color{new_turquoise}\insertenumlabel}
\setbeamertemplate{enumerate subitem}{\color{new_turquoise}\insertsubenumlabel}

\setbeamertemplate{caption}{\raggedright\insertcaption\par}

\setbeamercolor{section in toc}{fg=new_turquoise}
\setbeamercolor{subsection in toc}{fg=new_turquoise}
\setbeamercolor{subsubsection in toc}{fg=new_turquoise}






\usebackgroundtemplate{
    \includegraphics[width=\paperwidth,height=\paperheight]{figs/slide-title.jpg}
} 

\title{\fontsize{49}{7.2}{\bf Fundamental Algorithmic Techniques III}}
\date{\color{new_turquoise}\today}

\begin{document}
\frame{\titlepage}

\usebackgroundtemplate{
    \includegraphics[width=\paperwidth,height=\paperheight]{figs/slide-pages}
} 


\setbeamertemplate{subsection in toc}{
  \color{new_turquoise}$\blacksquare$\color{black}~~\inserttocsubsection
}


\begin{frame}{Outline}
    \tableofcontents
\end{frame}


\section{Divide \& Conquer}



\begin{frame}
    \frametitle{Multiplying Square Matrices}
    
 $$\mathbf{C} = \mathbf{A} \cdot \mathbf{B},$$
 $n \, \mathrm{operations} \, \forall i,\,j: \quad \quad \quad  c_{ij}= \sum_{k=0}^n a_{ik} \cdot b_{kj}.$

\bigskip

Divide and conquer:
\[
A = \begin{bmatrix}
A_{11} & A_{12} \\
A_{21} & A_{22}
\end{bmatrix},
\quad
B = \begin{bmatrix}
B_{11} & B_{12} \\
B_{21} & B_{22}
\end{bmatrix},
\quad
C = \begin{bmatrix}
C_{11} & C_{12} \\
C_{21} & C_{22}
\end{bmatrix},
\]
\[
\begin{bmatrix}
\textcolor{red}{C_{11}} & \textcolor{green}{C_{12}} \\
\textcolor{blue}{C_{21}} & \textcolor{yellow}{C_{22}}
\end{bmatrix}
=
\begin{bmatrix}
\textcolor{red}{A_{11}B_{11} + A_{12}B_{21}} & \textcolor{green}{A_{11}B_{12} + A_{12}B_{22}} \\
\textcolor{blue}{A_{21}B_{11} + A_{22}B_{21}} & \textcolor{yellow}{A_{21}B_{12} + A_{22}B_{22}}
\end{bmatrix}.
\tag{4.4}
\]

\bigskip

Decomposing with 8 sub-operations: $T(n) = 8 T(n/2) + \Theta(1)$, \\ 
so $\mathbf{T(n) = \Theta(n^3)}$ (master theorem 
$c = 3 = log_2(8)$).

\end{frame}


\begin{frame}
\frametitle{Strassen Algorithm}

$T(n) = 7 T(n/2) + \Theta(1)$,  
so $\mathbf{T(n) = \Theta(n^{2.81})}$


\footnotesize
\begin{align*}
M_1 &= (A_{11} + A_{22})(B_{11} + B_{22}) \\
M_2 &= (A_{21} + A_{22}) B_{11} \\
M_3 &= A_{11}(B_{12} - B_{22}) \\
M_4 &= A_{22}(B_{21} - B_{11}) \\
M_5 &= (A_{11} + A_{12}) B_{22} \\
M_6 &= (A_{21} - A_{11})(B_{11} + B_{12}) \\
M_7 &= (A_{12} - A_{22})(B_{21} + B_{22})
\end{align*}

\textbf{Result Blocks:}
\begin{align*}
C_{11} &= M_1 + M_4 - M_5 + M_7 \\
C_{12} &= M_3 + M_5 \\
C_{21} &= M_2 + M_4 \\
C_{22} &= M_1 - M_2 + M_3 + M_6
\end{align*}


\end{frame}



\begin{frame}
\frametitle{Find Maximum: Divide and Conquer}

\begin{columns}[T]
\column{0.5\textwidth}
\centering
\includegraphics[width=0.9\linewidth]{Algos_figs/max_divide_conquer.png} 

\column{0.5\textwidth}
\textbf{Problem:} 
Find the maximum element in an array of $n$ numbers.

\vspace{1em}

\textbf{Approach:}
\begin{itemize}
\item \textbf{Divide}: Split into two halves
    \item \textbf{Conquer}: Recursively find max
    \item \textbf{Combine}:  $\max(\text{left}, \text{right})$
\end{itemize}
\vspace{1em}
\vspace{1em}

\textbf{Complexity:} $T(n) = 2T(n/2) + cn$ \\ $\mathbf \Longrightarrow \mathcal{O}(n)$

\end{columns}

\end{frame}





\begin{frame}
    \frametitle{Sorting}

{\bf Insertion sort:} $T(n) = an^2 + bn + c$, $a,\,b\,c, \, \in \mathbb{N}$\\
So $T(N) = \mathcal{O}(n^2)$.


\bigskip
\bigskip

But can we do better? \\
Yes, actually $\mathcal{O}(n log(n))$ with MergeSort and QuickSort.
\begin{figure}
    \centering
    \includegraphics[width=0.5\linewidth]{Algos_figs/merge_sort_2.png}
    \caption{Merge Sort}
\end{figure}
    
\end{frame}



\section{Recurrence Relations}

\begin{frame}
    \frametitle{Recurrence Relation of D\&C: Mathematical Description}
    Let $ T(n) $ be a recurrence relation defined for $ n \geq 1 $ by:
\[
T(n) = a  T\left(\frac{n}{b}\right) + f(n)
\]
where:
\begin{itemize}

\item $ a \geq 1 $ is the number of subproblems in the recursion,
    \item $ b > 1 $ is the factor by which the input size is reduced in each subproblem,
    \item $ f(n) $ is the cost of dividing the problem and combining the results.
\end{itemize}

\end{frame}

\begin{frame}
\frametitle{Recursive: Not always good!}
\begin{table}[htbp]
\centering
\caption{Iterative vs Recursive Factorial: Complexity Comparison}
\begin{tabular}{lcc}
\toprule
 & \textbf{Iterative} & \textbf{Recursive} \\
\midrule
Time Complexity & $O(n)$ & $O(n)$ \\
Space Complexity & $O(1)$ & \textcolor{red}{$O(n)$} \\
Stack Overflow? & No & \textcolor{red}{Yes} \\
\bottomrule
\end{tabular}
\end{table}


\begin{figure}
    \centering
\includegraphics[width=0.4\linewidth]{Algos_figs/factorial_julia.png}
    \caption{Factorial in Julia}
\end{figure}

\end{frame}




\section{Master Theorem}

\begin{frame}
    \frametitle{Master Theorem}

\small


Asymptotic behavior of $T(n) = a \cdot T\left(\frac{n}{b}\right) + f(n)$:\\
\textbf{critical exponent}: $c_{\text{crit}} = \log_b a$

\footnotesize
\begin{enumerate}
    \item \textbf{Case 1 (Subproblem Dominated):} \\
    If $ f(n) = O(n^c) $ where $ c < c_{\text{crit}} $, then:
    \[
    T(n) = \Theta(n^{\log_b a})
    \]

    \item \textbf{Case 2 (Balanced):} \\
    If $ f(n) = \Theta(n^{c_{\text{crit}}}) $, then:
    \[
    T(n) = \Theta(n^{c_{\text{crit}}} \log n) = \Theta(n^{\log_b a} \log n)
    \]

    \item \textbf{Case 3 (Work Dominated):} \\
    If $ f(n) = \Omega(n^c) $ where $ c > c_{\text{crit}} $, and if the \textbf{regularity condition} holds:
    \[
    a  f\left(\frac{n}{b}\right) \leq k  f(n) \quad \text{for constant } k < 1 \text{ and all sufficiently large } n,
    \]
    then:
    \[
    T(n) = \Theta(f(n))
    \]
\end{enumerate}
    
\end{frame}


\begin{frame}
    \frametitle{Master Theorem : Limitations \&  Examples}

\small

{\bf Limitations:}
\small
\begin{itemize}
    \item $T(n)$ not monotone, e.g. $T(n) = sin (n)$
    \item $f(n)$ not polynomial, e.g. $f(n) = 2^n$
    \item $a$ not a constant, e.g. $a = 2n$
\end{itemize}

\bigskip

{\bf Examples:} verify!
\begin{enumerate}
    \item $T(n)=4T(\frac{n}{2})+n$, \\
        $ \Rightarrow \mathrm{Case} \, 1$, Subproblem dominated,         $T(n)=\Theta (n^2)$
    \item  $T(n)=2T(\frac{n}{2})+n$, \\
    $ \Rightarrow \mathrm{Case} \, 2$, Balanced,  $T(n)=\Theta(nlog(n))$
    \item $T(n) = 3T(\frac{n}{2}) + n^2$, \\
    $ \Rightarrow \mathrm{Case} \, 3$, Work dominated,  $T(n)=\Theta(n^2)$

    \item $T(n)=2T(2n)+nlog(n)$, \\
    $ \Rightarrow$  trivially not applicable!
    \item $T(n) = T(n-1) + 1$, \\
    $ \Rightarrow$ Not applicable!  $n-1\neq n/b$, actually $T(n)=\Theta(n!)$
\end{enumerate}    
    
\end{frame}



\begin{frame}
    \frametitle{Master Theorem : Proof with Tree Approach}

$$T(n)=aT(\frac{n}{b})+f(n)$$

\begin{itemize}
    \item level 0: work $f(n)$, $a$ subproblems of size $n/b$
    \item ...
    \item level i: work \( a^i \cdot f(n/b^i) \), \( a^i \) subproblems of size \( n/b^i \) 
\end{itemize}
stops when $i=log_b(n)$, and using $a \cdot log_b (n) = n\cdot log_b (a)$:

\[ T(n) = \textcolor{red}{\sum_{i=1}^{\log_b n} a^i \cdot f(n/b^i)} + \textcolor{green}{\Theta(n^{\log_b a})} ,\]
with \textcolor{red}{work} and \textcolor{green}{leaf decomposition} contributions.\\
\bigskip

...Study each 3 cases separately for the proof...

\end{frame}

\end{document}