

\documentclass{beamer}
\usepackage[utf8]{inputenc}
\usepackage{url}
\usepackage{algorithm}
\usepackage{algpseudocode}
\usepackage{lmodern}
\usepackage{natbib}
\usepackage{tikz}
\usepackage{physics}
\usepackage{graphicx} % Allows including images
\graphicspath{ {./images/} }
\usepackage{booktabs} % Allows the use of \toprule, \midrule and \bottomrule in tables
\usetikzlibrary{mindmap, trees, shadows, shapes, calc, fadings, positioning, decorations.pathreplacing, intersections, shapes, arrows}

%\usetheme{Hannover}
%\usecolortheme{spruce}
% EPIC FAIL
\usetheme{default}
%\usecolortheme{beetle}
\usepackage{graphics}

\usepackage{color}
\definecolor{new_turquoise}{RGB}{40,151,158}%{251,190,94}
\setbeamercolor{title}{fg=new_turquoise}


%\usepackage[colorlinks=true, urlcolor=blue, linkcolor=red]{hyperref}


\makeatletter
\setbeamertemplate{frametitle}{
    \ifbeamercolorempty[bg]{frametitle}{}{\nointerlineskip}%
    \@tempdima=\textwidth%
    \advance\@tempdima by\beamer@leftmargin%
    \advance\@tempdima by\beamer@rightmargin%
    \vspace*{0.8cm} 
    %\hspace*{-3cm}
    %%%%%%%%%%%%% For example insert shift to right
    \begin{beamercolorbox}[sep=0.3cm,center,wd=\the\@tempdima]{frametitle}
        \usebeamerfont{frametitle}%
        \vbox{}\vskip-1ex%
        \if@tempswa\else\csname beamer@ftecenter\endcsname\fi%
        \strut\insertframetitle\strut\par%
        {%
            \ifx\insertframesubtitle\@empty%
            \else%
            {\usebeamerfont{framesubtitle}\usebeamercolor[fg]{framesubtitle}\insertframesubtitle\strut\par}%
            \fi
        }%
        \vskip-1ex%
        \if@tempswa\else\vskip-.3cm\fi% set inside beamercolorbox... evil here...
    \end{beamercolorbox}%
}
\makeatother

\setbeamercolor{frametitle}{fg=new_turquoise}
\setbeamertemplate{itemize item}{\color{new_turquoise}$\blacksquare$}
\setbeamertemplate{itemize subitem}{\color{new_turquoise}$\blacksquare$}


% --- ENUMERATE ITEMS ---
\setbeamertemplate{enumerate item}{\color{new_turquoise}\insertenumlabel}
\setbeamertemplate{enumerate subitem}{\color{new_turquoise}\insertsubenumlabel}

\setbeamertemplate{caption}{\raggedright\insertcaption\par}

% --- COLOR THE TABLE OF CONTENTS ENTRIES ---
\setbeamercolor{section in toc}{fg=new_turquoise}
\setbeamercolor{subsection in toc}{fg=new_turquoise}
\setbeamercolor{subsubsection in toc}{fg=new_turquoise}



\usebackgroundtemplate{
    \includegraphics[width=\paperwidth,height=\paperheight]{figs/slide-title.jpg}
} 

\title{\fontsize{49}{7.2}{\bf Fundamental Algorithmic Techniques V}}
%\author{JW}
\date{\color{new_turquoise}\today}
%S\titlegraphic{\includegraphics[width=2cm]{figs/jw.png}}

\begin{document}
\frame{\titlepage}
%% SLIDE 1 - INTRO TO THE TEAM
%%%%%%%%%%%%%%%%%%%%%%%%%%%%%%%%%%%%%%%%%%%%%%%%%%%%%

\usebackgroundtemplate{
    \includegraphics[width=\paperwidth,height=\paperheight]{figs/slide-pages}
} 


\setbeamertemplate{subsection in toc}{
  \color{new_turquoise}$\blacksquare$\color{black}~~\inserttocsubsection
}


% Outline frame
\begin{frame}{Outline}
    \tableofcontents
\end{frame}



\section{The greedy algorithm paradigm} 
\begin{frame}{The greedy algorithm paradigm}
    %It’s fast and simple—but only correct for problems with the greedy choice property. 
Best possible (greedy) choice right now, for immediate best outcome! \\
\medskip 

Requirements:
\begin{enumerate}
    \item {\bf greedy-choice property:} \\  
    globally optimal solution $\Leftrightarrow$  local optimal (greedy) choices
    \item {\bf optimal substructure}
\end{enumerate}
\bigskip 
Examples where Greedy Algorithm is suboptimal
\begin{itemize}
    \item life!?
    \item road...
    \item 0-1 knapsack problem
\end{itemize}
    
\end{frame}


\begin{frame}{Examples with Greedy: Courses Allocation}
\begin{columns}[T] % align columns at the top
    \begin{column}{0.55\textwidth}
        \textbf{Course allocation:}\\
        For starting time $\mathrm{T}$:
        \begin{itemize}
        \item Select out courses with starting $< \mathrm{T}$
            \item Choose remaining course $C$ with lowest start time $T_{\text{end}}$
            \item Update $\mathrm{T} \leftarrow C_{T_{\text{end}}}$
        \end{itemize}
    \end{column}
    \begin{column}{0.4\textwidth}
        \centering
        \includegraphics[width=\linewidth]{Algos_figs/GreedyClasses.png}
    \end{column}
\end{columns}
\end{frame}



%\begin{frame}{Examples with Greedy:  %Offline Cache} 
%
%\end{frame}


\begin{frame}{Examples with Greedy: Cache Memory Management}
\begin{columns}[T]
    \begin{column}{0.6\textwidth}
        On request for block $b_i$:
        \begin{itemize}
        \item \textbf{Hit:} $b_i$ is in cache → no change.
            \item \textbf{Miss, cache not full:} add $b_i$.
            \item \textbf{Miss, cache full:} evicts one block, add $b_i$. 
        \end{itemize}

    \bigskip
    \bigskip

    Greedy Strategy for cache allocation \\
    removing less used cache blocks
    
    \end{column}
    \begin{column}{0.5\textwidth}
        \centering
        \includegraphics[width=\linewidth]{Algos_figs/cache.png}

        %\centering
        \includegraphics[width=\linewidth]{Algos_figs/fully_associative_mapping.png}

        
    \end{column}

    
\end{columns}
\end{frame}



\begin{frame}{Huffman Encoding Example}
\begin{columns}[T]
    \begin{column}{0.3\textwidth}
        \centering
        \includegraphics[width=\linewidth]{Algos_figs/HuffmanCodeAlg.png}
    \end{column}
    \begin{column}{0.7\textwidth}
    \centering
        \includegraphics[width=0.5\linewidth]{Algos_figs/Huffman_tree_2.svg.png}
        \centering
        \tiny
        \begin{tabular}{l|c|c}
            \textbf{Char} & \textbf{Freq} & \textbf{Code} \\
            \hline
            space & 7 & 111 \\
            a     & 4 & 010 \\
            e     & 4 & 000 \\
            f     & 3 & 1101 \\
            h     & 2 & 1010 \\
            i     & 2 & 1000 \\
            m     & 2 & 0111 \\
            n     & 2 & 0010 \\
            s     & 2 & 1011 \\
            t     & 2 & 0110 \\
            l     & 1 & 11001 \\
            o     & 1 & 00110 \\
            p     & 1 & 10011 \\
            r     & 1 & 11000 \\
            u     & 1 & 00111 \\
            x     & 1 & 10010 \\
        \end{tabular}
    \end{column}

\end{columns}
\end{frame}


\begin{frame}{Huffman Code: Numerical Example}
\tiny
\begin{tabular}{l|ccccc|c}
\textbf{Input } $(A, W)$ & \multicolumn{5}{c|}{\textbf{Symbol } ($a_i$)} & \\
                         & a & b & c & d & e & \textbf{Sum} \\ \hline
\textbf{Weights } ($w_i$) & 0.10 & 0.15 & 0.30 & 0.16 & 0.29 & $=1$ \\ \hline
\textbf{Output } $C$ & \multicolumn{5}{c|}{\textbf{Codewords } ($c_i$)} & \\
                      & 010 & 011 & 11 & 00 & 10 & \\ \hline
\textbf{Codeword length } ($\ell_i$) & 3 & 3 & 2 & 2 & 2 & \\ \hline
$\ell_i w_i$ & 0.30 & 0.45 & 0.60 & 0.32 & 0.58 & $L(C) = 2.25$ \\ \hline
\textbf{Optimality} & \multicolumn{5}{c|}{\textbf{Probability budget } ($2^{-\ell_i}$)} & \\
                    & $1/8$ & $1/8$ & $1/4$ & $1/4$ & $1/4$ & $=1.00$ \\ \hline
\textbf{Info. content } ($-\log_2 w_i$) & 3.32 & 2.74 & 1.74 & 2.64 & 1.79 & \\ \hline
$-w_i \log_2 w_i$ & 0.332 & 0.411 & 0.521 & 0.423 & 0.518 & $H(A) = 2.205$
\end{tabular}


\bigskip
\bigskip

Huffman coding approximates the optimal lossless compression bound!
\begin{itemize}
\item The Huffman code minimizes the expected length:
$
L(C) = \sum_{i} w_i \, \ell_i
$

\item The (Shannon) entropy of the source is:
$
H(A) = -\sum_{i} w_i \log_2 w_i
$

\item Huffman coding is near-optimal:
$
H(A) \leq L(C) < H(A) + 1
$
\end{itemize}

\end{frame}







%\begin{frame}{ Examples where Greedy is optimal} 
%Course allocation, Cache, Huffman coding and Shannon Entropy
%\end{frame}

%\begin{frame}{ Bellman ford} 
%\end{frame}




\section{Characteristics of greedy algorithms}



\begin{frame}{Characteristics of Greedy Algorithms}

In addition to greedy property and optimal substructures...
\bigskip
\begin{itemize}
    
  \item \textbf{A candidate set} – A solution is created from this set.
    \item \textbf{A selection function} – Used to choose the best candidate to be added to the solution.
    \item \textbf{A feasibility function} – Used to determine whether a candidate can be used to contribute to the solution.
    \item \textbf{An objective function} – Used to assign a value to a solution or a partial solution.
    \item \textbf{A solution function} – Used to indicate whether a complete solution has been reached.

\end{itemize}
\end{frame}

\section{Correctness proof techniques}

%\begin{frame}{Correctness Proof Techniques}
%{\bf Greedy stays ahead!}
%\begin{itemize}
%    \item $A = \{a_0, a_1, \dots, a_n \}$ greedy algo. sequence and  $O = \{o_0, o_1, \dots, o_n \}$ optimal sequence
%    \item choose a measure $\mu()$ or cost function.
%    \item show stay ahead:\\ $\mu(a_0, a_1, \dots, a_n ) \leq \mu(o_0, o_1, \dots, o_n )$ by induction! 
%    \item Prove optimality: use greedy stays ahead to generate contradiction
%\end{itemize}
%
%\end{frame}


\begin{frame}{Correctness Proof: Greedy Stays Ahead}
\textbf{Idea:} Show greedy solution is \textit{at least as far along} as optimal after each step.

\begin{itemize}
    \item Let $A = (a_1, \dots, a_k)$ be greedy solution, $O = (o_1, \dots, o_m)$ optimal.
    \item Define a \textbf{progress measure} $\pi(\cdot)$ (e.g., finish time, coverage).
    \item \textbf{Key claim (by induction):} After $i$ steps, 
    \[
    \pi(a_1, \dots, a_i) \geq \pi(o_1, \dots, o_i)
    \quad \text{(greedy is ``ahead'')}
    \]
    \item \textbf{Conclude optimality:} If greedy stays ahead for all $i$, then $k \geq m$. 
    Since $O$ is optimal, $k = m$ → $A$ is optimal.
\end{itemize}

\vspace{0.5em}
\bigskip
\footnotesize
Example: Interval scheduling (greedy picks earliest finish time).
\end{frame}



\end{document}


